
% vim: set sw=3 tw=79 expandtab:

\newcommand{\doctitle}{Position in Plane from Time of Circular Wavefront's
Arrival at Each of Three Detectors (Version 1.2)}

\documentclass[twocolumn]{article}

\usepackage{amsmath}        % for \text
\usepackage[english]{babel} % language selection
\usepackage{amsfonts}       % for \mathbb

% for formatting figure captions
\usepackage[margin=10pt,font={sf},labelfont=bf]%
           {caption}        % for formatting figure captions

\usepackage{fancyhdr}
\usepackage{graphicx}
\usepackage{lastpage}
\usepackage{natbib}
\usepackage{tikz}
\usepackage{times}
\usepackage{vmargin}
\usepackage{xcolor}

\usepackage[colorlinks=true,citecolor=blue,hyperfootnotes=false]
           {hyperref} % This must be the last package.

\selectlanguage{english} % Configure babel.

% vmargin setup
\setpapersize{USletter}
\setmarginsrb%
{0.375in}%           left
{0.375in}%           top
{0.375in}%           right
{0.5in}%             bottom
{2\baselineskip}%    headheight
{2\baselineskip}%    headsep
{3\baselineskip}%    footheight
{4\baselineskip}%    footskip

% mydate macro
\newcommand{\mydate}{%
   \number\year\space%
   \ifcase\month\or%
      Jan\or\ Feb\or\ Mar\or\ Apr\or\ May\or\ Jun\or%
      Jul\or\ Aug\or\ Sep\or\ Oct\or\ Nov\or\ Dec
   \fi\space%
   \number\day%
}

% fancyhdr settings
\pagestyle{fancy}
\lhead{\sffamily\textbf{\doctitle}}
\chead{}
\rhead{\sffamily \thepage~of~\pageref{LastPage}}
\renewcommand{\headrulewidth}{1pt}
\renewcommand{\footrulewidth}{1pt}
\lfoot{%
   \footnotesize\sffamily
   \begin{minipage}{0.95\textwidth}
   Copyright\ \copyright\ \ 2017\ \ Thomas E.\ Vaughan.
   PDF image generated on \mydate.
   Permission is granted to copy, distribute and/or modify this document under
   the terms of the GNU Free Documentation License, Version 1.3 or any later
   version published by the Free Software Foundation; with no Invariant
   Sections, no Front-Cover Texts, and no Back-Cover Texts.  A copy of the
   license is included in the section entitled ``GNU Free Documentation
   License''.
   \end{minipage}%
}
\cfoot{}
\rfoot{%
   \begin{minipage}{0.05\textwidth}
   \begin{flushright}
   \includegraphics[width=0.85\textwidth]{logo}
   \end{flushright}
   \end{minipage}%
}

\begin{document}

\thispagestyle{fancy}

\begin{abstract}
   If the radius of a circular wavefront, whose center is at coordinates
   $(x_0,y_0)$ in the plane, increase with constant speed $v$, then an array of
   three detectors, arranged in a right triangle, can be used to find $x_0$ and
   $y_0$.  If we choose the origin to be located at the first detector, which
   is at the right angle; if we choose the second detector to be located at
   $(x_2, 0)$; if we choose the third detector to be located at $(0, y_3)$; if
   the wavefront arrive at the second detector at a time difference $t_2$ from
   the time of arrival at the first detector; if the wavefront arrive at the
   third detector at a time difference $t_3$ from the time of arrival at the
   first detector; then the solution is
   \begin{equation*}
      (x_0, y_0) = \frac{(\beta_2 x_2, \beta_3 x_3)}{2} + (\alpha_2, \alpha_3)
                   \; \tau_0,
   \end{equation*}
   where
   \begin{eqnarray*}
      \tau_0 &=& \frac{\sqrt{\beta_2 \beta_3 \left[x_2^2 + y_3^2 - \left[\tau_2
                 - \tau_3\right]^2\right]} - \beta_2 \tau_2 - \beta_3 \tau_3}{2
                 \left[1 - \beta_2 - \beta_3\right]}\\[5pt]
      \tau_2 &=& vt_2\\
      \tau_3 &=& vt_3\\[5pt]
      \alpha_2 &=& \frac{\tau_2}{x_2}\\[5pt]
      \alpha_3 &=& \frac{\tau_3}{x_3}\\[5pt]
      \beta_2 &=& 1 - \alpha_2^2\\[5pt]
      \beta_3 &=& 1 - \alpha_3^2.
   \end{eqnarray*}
\end{abstract}

\section{Introduction}

A friend of mine wants to know how to take data from three acoustic sensors and
to calculate the location of a touch-point on a surface.  The idea is that
touching a surface causes sound waves to propagate from the touch-point. If
each acoustic sensor register the time of the leading edge of the sonic
disturbance, then it should be possible to calculate the location of the
touch-point.  The solution, however, seems hard to find in closed form unless
the detectors be arranged in a right triangle.

\section{Problem}

At each vertex of a right triangle, a detector records the time of a circular
wavefront's arrival.  The times refer to the same clock.  The wavefront expands
with constant speed $v$ from a center.  Find the coordinates of the center in
terms of the recorded times and the coordinates of the detectors.

\section{Coordinate System}

We must choose a coordinate system in which to solve the problem of converting
the times of arrival into the coordinates $(x_0, y_0)$ of the center of the
wavefront.

Let us locate the spatial origin and the first detector at the right angle of
the triangle, and let us locate the temporal origin at the time of the
wavefront's arrival at the first detector.  So the first detector has
coordinates $(x_1, y_1) = (0, 0)$, and the wavefront arrives there at time~$t_1
= 0$.

Let us choose the direction from the first detector toward the second
as the direction of the positive $x$~axis.  So the second detector has
coordinates $(x_2, y_2) = (x_2, 0)$, and the wavefront arrives there at
time~$t_2$.  Because of our choice, $x_2 > 0$.

Finally, the third detector has coordinates $(x_3, y_3) = (0, y_3)$, and the
wavefront arrives there at time~$t_3$.  Let us choose the direction of the
$y$~axis so that $y_3 > 0$.

\section{Solution}

We consider the arrival of the wavefront at each detector in turn.  In each
case,
\begin{equation}
   \left[x_i - x_0\right]^2 + \left[y_i - y_0\right]^2 = v^2 \left[t_i -
   t_0\right]^2,
\end{equation}
where $i \in \{1, 2, 3\}$, and $v$ is the speed of the wavefront.  This
indicates the relationships among the known coordinates $(x_i, y_i)$ of the
$i$th detector, the known time of arrival $t_i$ at that detector, the unknown
coordinates $(x_0, y_0)$ of the center of the wavefront, and the unknown time
$t_0$ at which the wavefront began its radiation from the center.  There are
three independent equations, one for each detector, and three unknown values.

\subsection{Wavefront Displacement}

Let us simplify our notation a bit by a change of variables.  Let us express
the signed distance traveled by the wavefront, between time~0 and any of the
four given times, as follows:
\begin{eqnarray}
   \tau_0 &=& vt_0\\
   \tau_1 &=& vt_1 = 0\\
   \tau_2 &=& vt_2\\
   \tau_3 &=& vt_3
\end{eqnarray}
Let us call each of these a \emph{wavefront displacement}.

The wavefront displacement $\tau_0$ is negative because $t_0$ is negative;
$\tau_0$ corresponds to the reverse motion of the wavefront from its crossing
of the location of the first detector and backward in time toward the
wavefront's center.  The distance $-\tau_0$ is the radial distance between the
center and the first detector.

Because each wavefront displacement has, as its starting point, the radial size
of the wavefront as it crosses the first detector and, as its ending point, the
radial size as the wavefront crosses the point indicated by the subscript
(either the number of a detector or else zero for the wavefront's center),
$\tau_1$ is zero by definition.  That is, the change in the radial size of the
wavefront from the time when it crosses the first detector until it crosses the
\emph{same} detector is just zero because the start and the end of that
displacement are the same.

If the wavefront reach the first detector before the second, then $\tau_2$ is
the radial distance traveled by the wavefront after it has passed the first
detector and until it has reached the second.  A negative value indicates that
the wavefront reached the second detector before the first.  In that case,
$-\tau_2$ is the radial distance traveled by the wavefront after it has passed
the second detector and until it has reached the first.

If the wavefront reach the first detector before the third, then $\tau_3$ is
the radial distance traveled by the wavefront after it has passed the first
detector and until it has reached the third.  A negative value indicates that
the wavefront reached the third detector before the first.  In that case,
$-\tau_3$ is the radial distance traveled by the wavefront after it has passed
the third detector and until it has reached the first.

In terms of wavefront displacements our fundamental equation becomes
\begin{equation}
   \left[x_i - x_0\right]^2 + \left[y_i - y_0\right]^2 = \left[\tau_i -
   \tau_0\right]^2.
\end{equation}
The unknowns are $x_0$, $y_0$, and $\tau_0$. Everything else is known, and the
knowns allow us to write down three equations, enough to solve for the
unknowns.

\subsection{Arrival at First Detector}

The equation at the first detector is the simplest because each coordinate of
the detector is zero, and the wavefront displacement at the first detector is
zero by definition.
\begin{equation}
   x_0^2 + y_0^2 = \tau_0^2
   \label{eq:first}
\end{equation}

\subsection{Arrival at Second Detector}

The equation at the second detector is more complicated than the equation at
the first.  Among the three known values at the detector, only the
$y$~coordinate of the detector is zero.
\begin{equation}
   \left[x_2 - x_0\right]^2 + y_0^2 = \left[\tau_2 - \tau_0\right]^2
   \label{eq:second}
\end{equation}
Subtracting the Equation~\ref{eq:first} from Equation~\ref{eq:second}, we find
\begin{eqnarray}
   \nonumber
   x_2^2 - 2 x_2 x_0 &=& \tau_2^2 - 2 \tau_2 \tau_0\\
   \nonumber
   2 x_2 x_0 &=& x_2^2 - \tau_2^2 + 2 \tau_2 \tau_0\\
   x_0 &=& \frac{1}{2}\left[1 - \frac{\tau_2^2}{x_2^2}\right] x_2 +
           \left[\frac{\tau_2}{x_2}\right] \tau_0.
\end{eqnarray}
Here we introduce for convenience a dimensionless, known value,
\begin{equation}
   \alpha_2 = \frac{\tau_2}{x_2} = \frac{v t_2}{x_2},
\end{equation}
which is the ratio between the wavefront displacement $\tau_2 = vt_2$ and the
distance $x_2$ from the first detector to the second.  This ratio is constrained
so that
\begin{equation}
   -1 \leq \alpha_2 \leq 1.
\end{equation}
In terms of $\alpha_2$, we have
\begin{equation}
   x_0 = \frac{1}{2}\left[x_2 - \alpha_2\tau_2\right] + \alpha_2\tau_0.
   \label{eq:t-x}
\end{equation}

\subsection{Arrival at Third Detector}

The equation at the third detector is as complicated as
Equation~\ref{eq:second} but has a different form.
\begin{equation}
   x_0^2 + \left[y_3 - y_0\right]^2 = \left[\tau_3 - \tau_0\right]^2
   \label{eq:third}
\end{equation}
Subtracting Equation~\ref{eq:first} from Equation~\ref{eq:third}, we find
\begin{equation}
   y_0 = \frac{1}{2}\left[1 - \frac{\tau_3^2}{y_3^2}\right] y_3 +
         \left[\frac{\tau_3}{y_3}\right] \tau_0.
\end{equation}
Here, too, we introduce a dimensionless, known value,
\begin{equation}
   \alpha_3 = \frac{\tau_3}{y_3} = \frac{v t_3}{y_3},
\end{equation}
which is the ratio between the wavefront displacement $\tau_3 = vt_3$ and the
distance $y_3$ from the first detector to the third.  Like $\alpha_2$, the
ratio $\alpha_3$ is constrained so that
\begin{equation}
   -1 \leq \alpha_3 \leq 1.
\end{equation}
In terms of this ratio, we have
\begin{equation}
   y_0 = \frac{1}{2}\left[y_3 - \alpha_3\tau_3\right] + \alpha_3\tau_0.
   \label{eq:t-y}
\end{equation}

\subsection{Solving for $\tau_0$}

Now we are in position to bring everything together. Substituting
Equations~\ref{eq:t-x} and~\ref{eq:t-y} into Equation~\ref{eq:first}, we find
\begin{equation}
   a\tau_0^2 + b\tau_0 + c = 0,
\end{equation}
where
\begin{eqnarray}
   a &=& \alpha_2^2 + \alpha_3^2 - 1\\[5pt]
   b &=& \left[1 - \alpha_2^2\right]\tau_2 + \left[1 -
         \alpha_3^2\right]\tau_3\\[5pt]
   c &=& \frac{x_2^2 + \alpha_2^2\tau_2^2 + y_3^2 + \alpha_3^2\tau_3^2}{4} -
         \frac{\tau_2^2 + \tau_3^2}{2}.
\end{eqnarray}
In order to use the quadratic formula, we first evaluate
\begin{equation}
   b^2 = \left[1 - \alpha_2^2\right]^2 \tau_2^2 + 2 \left[1 -
   \alpha_2^2\right]\left[1 - \alpha_3^2\right] \tau_2 \tau_3 + \left[1 -
   \alpha_3^2\right]^2 \tau_3^2.
\end{equation}
Next, we evaluate
\begin{equation}
   4ac = \left[\alpha_2^2 + \alpha_3^2 - 1\right] \left[x_2^2 + \alpha_2^2
   \tau_2^2 + y_3^2 + \alpha_3^2 \tau_3^2 - 2\left[\tau_2^2 +
   \tau_3^2\right]\right],
\end{equation}
which we can write as
\begin{eqnarray}
   \nonumber
   4ac &=&
           \tau_2^2 + \alpha_2^4 \tau_2^2 + \alpha_2^2 y_3^2 + \alpha_2^2
           \alpha_3^2 \tau_3^2 - 2 \alpha_2^2 \left[\tau_2^2 + \tau_3^2\right]
           +\\
   \nonumber
   &&
           \alpha_3^2 x_2^2 + \alpha_3^2 \alpha_2^2 \tau_2^2 + \tau_3^2 +
           \alpha_3^4 \tau_3^2 - 2 \alpha_3^2 \left[\tau_2^2 + \tau_3^2\right]
           +\\
   \nonumber
   &&
           2 \left[\tau_2^2 + \tau_3^2\right] - x_2^2 - \alpha_2^2 \tau_2^2 -
           y_3^2 - \alpha_3^2 \tau_3^2\\[5pt]
   \nonumber
   &=&
           \alpha_2^4 \tau_2^2 + \alpha_2^2 \alpha_3^2 \tau_3^2 - 2 \alpha_2^2
           \tau_3^2 - \left[1 - \alpha_3^2\right] x_2^2 +\\
   \nonumber
   &&
           \alpha_3^4 \tau_3^2 + \alpha_3^2 \alpha_2^2 \tau_2^2 - 2 \alpha_3^2
           \tau_2^2 - \left[1 - \alpha_2^2\right] y_3^2 +\\
   \nonumber
   &&
           3 \left[\left[1 - \alpha_2^2\right]\tau_2^2 + \left[1 -
           \alpha_3^2\right] \tau_3^2\right]\\[5pt]
   \nonumber
   4ac &=&
           \alpha_2^4 \tau_2^2 + \left[1 - \alpha_2^2\right] \left[1 -
           \alpha_3^2\right] \tau_3^2 - \alpha_2^2\tau_3^2 - \left[1 -
           \alpha_3^2\right] x_2^2 +\\
   \nonumber
   &&
           \alpha_3^4 \tau_3^2 + \left[1 - \alpha_3^2\right] \left[1 -
           \alpha_2^2\right] \tau_2^2 - \alpha_3^2 \tau_2^2 - \left[1 -
           \alpha_2^2\right] y_3^2 +\\
   &&
           2 \left[\left[1 - \alpha_2^2\right]\tau_2^2 + \left[1 -
           \alpha_3^2\right] \tau_3^2\right].
\end{eqnarray}
Next, we evaluate
\begin{eqnarray}
   \nonumber
   b^2 - 4ac &=& - \left[1 - \alpha_2^2\right] \left[1 - \alpha_3^2\right]
                   \left[\tau_2 - \tau_3\right]^2\\
   \nonumber
              && + \left[1 - \alpha_3^2\right] \left[x_2^2 - \tau_2^2\right]\\
   \nonumber
              && + \left[1 - \alpha_2^2\right] \left[y_3^2 -
                 \tau_3^2\right]\\[5pt]
   b^2 - 4ac &=& \beta_2 \beta_3 \left[x_2^2 + y_3^2 - \left[\tau_2 -
                 \tau_3\right]^2\right],
\end{eqnarray}
where
\begin{eqnarray}
   \beta_2 &=& 1 - \alpha_2^2\\[5pt]
   \beta_3 &=& 1 - \alpha_3^2.
\end{eqnarray}
Finally, we have
\begin{equation}
   \tau_0 = \frac{ \sqrt{\beta_2 \beta_3 \left[x_2^2 + y_3^2 - \left[\tau_2 -
            \tau_3\right]^2\right]} - \beta_2 \tau_2 - \beta_3 \tau_3 }{2
            \left[1 - \beta_2 - \beta_3\right]}.
   \label{eq:t}
\end{equation}
We must select the positive branch for the square root because $\tau_0 < 0$.
We can see that the form is correct by studying a special case.  Consider the
case in which $(x_0, y_0)$ is located so that the wavefront arrives at all
three detectors simultaneously.  In that case, $\tau_2 = \tau_3 = 0$, and
$\beta_2 = \beta_3 = 1$.  The denominator becomes negative, and the terms
outside the root become zero; so we much use the positive root in order for
$\tau_0$ to be negative.  If we specify the same case further, so that $x_2 =
y_3 = a$, then we find $\tau_0 = -\frac{a}{\sqrt{2}}$, just as we should expect
when the center of the wavefront is located in the center of the square, each
of three of whose vertices is occupied by a detector.

\subsection{Solving for $x_0$ and $y_0$}

Equations~\ref{eq:t-x} and~\ref{eq:t-y} can be rewritten as
\begin{equation}
   (x_0, y_0) = \frac{(\beta_2 x_2, \beta_3 x_3)}{2} + (\alpha_2, \alpha_3) \;
                \tau_0,
\end{equation}
where $\tau_0$ is given by Equation~\ref{eq:t}.

\section{Conclusion}

The location of the center of the wavefront has been determined in terms of the
relative timing of the arrival of the wavefront at each of three sensors.

A logical next step is to calculate how a fixed error in the timing measurement
would affect the accuracy of the estimate of the location of the center.  In
general, the accuracy should degrade as the center moves sufficiently far away
from the array of detectors.

A future release of the present paper may include an analysis of this and other
likely errors in any realistic implementation.

%\bibliographystyle{plainnat}
%
%\begin{thebibliography}{}
%
%\bibitem[LastName1 et al.(Year)LastName1, LastName2, and LastName3]{CiteTag}
%LastName1, Initials1, Initials2 LastName2, and Initials3 LastName3\ \
%``Title of Article''\ \ {\it Title of Journal}, {\bf Volume}: Pages, Year.
%
%\end{thebibliography}

\newpage

\input{fdl-1.3}

\end{document}

