
\newcommand{\doctitle}{Position in Plane from Time of Circular Wavefront's
Arrival at Each of Three Detectors}

\documentclass[twocolumn]{article}

\usepackage{amsmath}        % for \text
\usepackage[english]{babel} % language selection
\usepackage{amsfonts}       % for \mathbb

% for formatting figure captions
\usepackage[margin=10pt,font={sf},labelfont=bf]%
           {caption}        % for formatting figure captions

\usepackage{fancyhdr}
\usepackage{graphicx}
\usepackage{lastpage}
\usepackage{natbib}
\usepackage{tikz}
\usepackage{times}
\usepackage{vmargin}
\usepackage{xcolor}

\usepackage[colorlinks=true,citecolor=blue,hyperfootnotes=false]
           {hyperref} % This must be the last package.

\selectlanguage{english} % Configure babel.

% vmargin setup
\setpapersize{USletter}
\setmarginsrb%
{0.375in}%           left
{0.375in}%           top
{0.375in}%           right
{0.5in}%             bottom
{2\baselineskip}%    headheight
{2\baselineskip}%    headsep
{3\baselineskip}%    footheight
{4\baselineskip}%    footskip

% mydate macro
\newcommand{\mydate}{%
   \number\year\space%
   \ifcase\month\or%
      Jan\or\ Feb\or\ Mar\or\ Apr\or\ May\or\ Jun\or%
      Jul\or\ Aug\or\ Sep\or\ Oct\or\ Nov\or\ Dec
   \fi\space%
   \number\day%
}

% fancyhdr settings
\pagestyle{fancy}
\lhead{\sffamily\textbf{\doctitle}}
\chead{}
\rhead{\sffamily \thepage~of~\pageref{LastPage}}
\renewcommand{\headrulewidth}{1pt}
\renewcommand{\footrulewidth}{1pt}
\lfoot{%
   \footnotesize\sffamily
   \begin{minipage}{0.95\textwidth}
   Copyright\ \copyright\ \ 2014\ \ Thomas E.\ Vaughan.
   PDF image generated on \mydate.
   Permission is granted to copy, distribute and/or modify this document under
   the terms of the GNU Free Documentation License, Version 1.3 or any later
   version published by the Free Software Foundation; with no Invariant
   Sections, no Front-Cover Texts, and no Back-Cover Texts.  A copy of the
   license is included in the section entitled ``GNU Free Documentation
   License''.
   \end{minipage}%
}
\cfoot{}
\rfoot{%
   \begin{minipage}{0.05\textwidth}
   \begin{flushright}
   \includegraphics[width=0.85\textwidth]{logo}
   \end{flushright}
   \end{minipage}%
}

\begin{document}

\thispagestyle{fancy}

\begin{abstract}

   If the radius of a circular wavefront, centered on a point $p$ at
   coordinates $(x_0,y_0)$ in the plane, increase with constant speed $v$, then
   an array of three detectors, arranged in a triangle of nonzero area, can be
   used to find $x_0$ and $y_0$.

\end{abstract}

\section{Introduction}

An array of three sensors, each of which records the time of the passage of the
front of an expanding wave, can be used to locate the epicenter of the wave.
The solution, however, is hard to find in closed form unless the sensors be
arranged in a right triangle.

\section{Problem}

At each vertex of a right triangle, a detector records the time of a circular
wavefront's arrival.  The times refer to the same clock.  The wavefront expands
with constant speed $v$ from a central point.  Find the location of the point.

\section{Coordinate System}

We must choose a coordinate system in which to solve the problem of converting
the times of arrival into the coordinates $(x_0, y_0)$ of the center of the
wavefront.

Let us place the spatial origin at the first detector, which is situated at the
right angle of the triangle, and the temporal origin at the time of the
wavefront's arrival at that detector.  So the first detector has coordinates
$(x_1, y_1) = (0, 0)$, and the wavefront arrives there at time~$t_1 = 0$.

Suppose that the second detector is located at a nonzero distance from the
first.  Let us choose the direction from the first detector toward the second
as the direction of the positive $x$~axis.  So the second detector has
coordinates $(x_2, y_2) = (x_2, 0)$, and the wavefront arrives there at
time~$t_2$.  Because of our choice, $x_2 > 0$.

Finally, the third detector has coordinates $(0, y_3)$, and the wavefront
arrives there at time~$t_3$.  Let us choose the direction of the $y$~axis so
that $y_3 > 0$.

\section{Solution}

We consider the arrival of the wavefront at each detector in turn.  In each
case,
\begin{equation}
   [x_i - x_0]^2 + [y_i - y_0]^2 = v^2 [t_i - t_0]^2,
\end{equation}
where $i \in \{1, 2, 3\}$, and $v$ is the speed of the wavefront.  This
indicates the relationship between the known coordinates $(x_i, y_i)$ of the
$i$th detector, the known time of arrival $t_i$ at that detector, the unknown
coordinates $(x_0, y_0)$ of the center of the wavefront, and the unknown time
$t_0$ at which the wavefront began its radiation from that point.  There are
three independent equations, one for each detector, and three unknown values.

Let us simplify our notation a bit by a change of variables.  Let us express
the signed distance traveled by the wavefront, between time~0 and any of the
four given times, as follows:
\begin{eqnarray}
   \tau_0 &=& vt_0\\
   \tau_1 &=& vt_1 = 0\\
   \tau_2 &=& vt_2\\
   \tau_3 &=& vt_3
\end{eqnarray}
Let us call each of these a \emph{wavefront displacement}.  Then our
fundamental equation becomes
\begin{equation}
   [x_i - x_0]^2 + [y_i - y_0]^2 = [\tau_i - \tau_0]^2.
\end{equation}
The unknowns are $x_0$, $y_0$, and $\tau_0$. Everything else is known, and the
knowns allow us to write down three equations, enough to solve for the
unknowns.

\subsection{Arrival at First Detector}

The equation at the first detector is the simplest because each coordinate of
the detector is zero, and the wavefront displacement at the first detector is
zero by definition.
\begin{equation}
   x_0^2 + y_0^2 = \tau_0^2
   \label{eq:first}
\end{equation}

\subsection{Arrival at Second Detector}

The equation at the second detector is more complicated than the equation at
the first.  Among the three known values at the detector, only the
$y$~coordinate of the detector is zero.
\begin{equation}
   [x_2 - x_0]^2 + y_0^2 = [\tau_2 - \tau_0]^2
   \label{eq:second}
\end{equation}
Subtracting the Equation~\ref{eq:first} from Equation~\ref{eq:second}, we find
\begin{eqnarray}
   \nonumber x_2^2 - 2 x_2 x_0 &=& \tau_2^2 - 2 \tau_2 \tau_0\\
   \nonumber 2 \tau_2 \tau_0   &=& \tau_2^2 - x_2^2 + 2 x_2 x_0\\
   \tau_0 &=&
      \frac{1}{\tau_2} \left[\frac{\tau_2^2 - x_2^2}{2} + x_2 x_0\right].
   \label{eq:t-x}
\end{eqnarray}

\subsection{Arrival at Third Detector}

The equation at the third detector is as complicated as
Equation~\ref{eq:second} but has a different form.
\begin{equation}
   x_0^2 + [y_3 - y_0]^2 = [\tau_3 - \tau_0]^2
   \label{eq:third}
\end{equation}
Subtracting Equation~\ref{eq:first} from Equation~\ref{eq:third}, we find
\begin{equation}
   \tau_0 = \frac{1}{\tau_3} \left[ \frac{\tau_3^2 - y_3^2}{2} + y_3
   y_0\right].
   \label{eq:t-y}
\end{equation}

\subsection{Eliminating $\tau_0$}

Setting the right-hand side of Equation~\ref{eq:t-x} equal to the right-hand
side of Equation~\ref{eq:t-y}, we find
\begin{equation}
   \frac{\tau_3^2 - y_3^2}{2} + y_3 y_0 = \frac{\tau_3}{\tau_2}
   \left[\frac{\tau_2^2 - x_2^2}{2} + x_2 x_0\right],
\end{equation}
which results in
\begin{equation}
   y_0 = \frac{1}{y_3} \left[ \frac{\tau_3}{\tau_2} \left[ \frac{\tau_2^2 -
   x_2^2}{2} + x_2 x_0\right] - \frac{\tau_3^2 - y_3^2}{2} \right].
\end{equation}

%\bibliographystyle{plainnat}
%
%\begin{thebibliography}{}
%
%\bibitem[LastName1 et al.(Year)LastName1, LastName2, and LastName3]{CiteTag}
%LastName1, Initials1, Initials2 LastName2, and Initials3 LastName3\ \
%``Title of Article''\ \ {\it Title of Journal}, {\bf Volume}: Pages, Year.
%
%\end{thebibliography}

\newpage

\input{fdl-1.3}

\end{document}

